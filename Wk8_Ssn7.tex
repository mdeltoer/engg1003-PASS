\documentclass{pass}

\usepackage{graphicx}
\usepackage{float}
\usepackage{soul}
\usepackage{enumitem}
\usepackage{tabularx}
\usepackage{amsmath,amsfonts,amssymb} % For nice maths

\title{ENGG1003 - PASS Session 7}
\author{Mitchell Deltoer}
\date{Week 8}

\begin{document}
\begin{minipage}[H]{0.6\textwidth}
  \maketitle
\end{minipage}
\hfill
\begin{minipage}[H]{0.35\textwidth}
\textbf{
\begin{tabular}{ccc}
Monday & 14-15 & ES238 \\
Wednesday & 12-13 & ES238 \\
Thursday & 10-11 & MCLG42 \\
\end{tabular}
}
\end{minipage}

\section*{Pointers}
A \textit{pointer} is a variable whose value is the address of another variable. The syntax for declaring a pointer variable is
\begin{lstlisting}[style=CStyle]
	data_type *var_ptr;
\end{lstlisting}
where \texttt{data\_type} is the base data type which the pointer is pointing at and \texttt{var\_ptr} is the name of the pointer variable.

You have already seen \textit{pointers} before. Using a variable name with an ampersand (\texttt{\&}) before it accesses the address of the variable rather than the value that it holds. So an example pointer declaration would look something like
\begin{lstlisting}[style=CStyle]
	int var;					// declaring a integer variable
	int *var_ptr = &var;	// initialising a pointer-to-int with the address of var
\end{lstlisting}

The most common use for \textit{pointers} in the context of this course is using them in functions.


\section*{Pointers and Functions}

This relates to arrays in that \textbf{the name of an array without an index is a \textit{pointer} to the first value in the array}. That is, the following two \texttt{printf} statements of code are equivalent:
\begin{lstlisting}[style=CStyle]
	int array[2];		// declaring an array of int, length 2
	printf("This is the address of array[0]: %lu\n", array);
	printf("This is the address of array[0]: %lu\n", &array[0]);
\end{lstlisting}

When using arrays in functions, a pointer to the first value in the array and the array size are both required as separate arguments (inputs) to the function. Because \textbf{the function is using the address of the original array, when values in the array are changed in the function they are also updated in the main function}.

\begin{lstlisting}[style=CStyle]
	// either one of these is a valid function prototype
	void example_function(int *array_name, int array_size)	
	void example_function(int array_name[], int array_size)
\end{lstlisting}


\pagebreak

\section*{Practice Programming}
Some things to consider when using functions within your program:
\begin{itemize}
\item What \textit{arguments} (inputs) are there to the function? What are their data types?
\item What is the \textit{return value} (output) of the function? What is its data type?
\item Have I remembered to (correctly) include the \textit{function prototype} and the \textit{function definition}?
\end{itemize}

\begin{task}{Set array values to zero}{}
	\begin{enumerate}
	\item Write a C program to set all the values within an array to zero. Print the entire array to confirm your result.\\[6pt]
	\item Modify the previous program to instead use a \textit{function} to set all the values to zero. The array should still be declared from within the main function.
	\end{enumerate}
\end{task}

\end{document}