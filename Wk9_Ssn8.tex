\documentclass{pass}

\usepackage{graphicx}
\usepackage{float}
\usepackage{soul}
\usepackage{enumitem}
\usepackage{tabularx}
\usepackage{amsmath,amsfonts,amssymb} % For nice maths

\title{ENGG1003 - PASS Session 8}
\author{Mitchell Deltoer}
\date{Week 9}

\begin{document}
\begin{minipage}[H]{0.6\textwidth}
  \maketitle
\end{minipage}
\hfill
\begin{minipage}[H]{0.35\textwidth}
\textbf{
\begin{tabular}{ccc}
Monday & 14-15 & ES238 \\
Wednesday & 12-13 & ES238 \\
Thursday & 10-11 & MCLG42 \\
\end{tabular}
}
\end{minipage}

\section*{Multi-dimensional arrays}
The simplest form of multidimensional array is the two-dimensional array. In essence a 2D array is a list of one-dimensional arrays. The following is an example of declaring a 2D array of size \texttt{[x][y]}: 
\begin{lstlisting}[style=CStyle]
	type arrayName [x][y];
\end{lstlisting}
where type is the array data type and arrayName is a valid C variable name. A 2D array can be considered as a table which will have \texttt{x} number of rows and \texttt{y} number of columns. A 2D array \texttt{a}, which contains three rows and four columns can be shown as follows
\begin{figure}[H]
    \begin{center}
        \includegraphics[width=0.5\textwidth]{table.jpg}
    \end{center}
\end{figure}
Thus, every element in the array \texttt{a} is identified by an element name of the form \texttt{a[i][j]}, where \texttt{a} is the name of the array, and \texttt{a} and \texttt{a} are the subscripts that uniquely identify each element in \texttt{a}.

Multidimensional arrays may be initialized by specifying bracketed values for each row. The following initialises an array with 3 rows and each row has 4 columns
\begin{lstlisting}[style=CStyle]
	int a[3][4] = {  
		{0, 1, 2, 3} ,   // initializers for row indexed by 0
		{4, 5, 6, 7} ,   // initializers for row indexed by 1
		{8, 9, 10, 11}   // initializers for row indexed by 2
	};
\end{lstlisting}
The line breaks in the above code are just for readability and are not required.

Just like with 1D arrays, an element in a 2D array is accessed by using the array indexes like so
\begin{lstlisting}[style=CStyle]
	int val = a[2][3];	// store the value 11 from the array in val
\end{lstlisting}

Like with 1D arrays, loops will be needed if you want to access each element of a 2D array. However, an additional nested loop is required for each additional dimension of the array, i.e. one nested loop for a 2D array. The following is an example of printing each value of the 2D array \texttt{a}
\begin{lstlisting}[style=CStyle]
   int i, j;
   for ( i = 0; i < 3; i++ ) {		// loop for the row index
      for ( j = 0; j < 4; j++ ) {	// loop for the column index
         printf("a[%d][%d] = %d\n", i,j, a[i][j] );
      }
   }
\end{lstlisting}

\pagebreak

\section*{Practice Programming}
\begin{task}{2D Array Sum}{}
Write a C program that finds the sum of all of the elements in a 2D array. Use the following initialisation for your initial testing.
\begin{lstlisting}[style=CStyle]
	int a[3][4] = {  
		{0, 1, 2, 3} ,
		{4, 5, 6, 7} ,   
		{8, 9, 10, 11}   
	};
\end{lstlisting}

\end{task}

\begin{task}{Row Sum}{}
Write a C program that finds the smallest number in each row of a 2D array. The result for each \textit{ith} row should be stored in the \textit{ith} element of a 1D array. Use the same initialisation as previous for initial testing.

\end{task}

\begin{task}{2D Matrix Inverse}{}
\textbf{Note:} You won't have seen these words or concepts before but it's not necessary to complete this task.\\

For a $2\times2$ matrix (2D array: 2 rows, 2 columns) $A$ defined as
$$ A = \begin{bmatrix} a & b \\ c & d \end{bmatrix} $$
the \textit{determinant} of the matrix $A$ can be calculated by
$$ \text{det}(A) = ad-bc $$
If the determinant is zero, the matrix is said to be \textit{non-invertible}, i.e. its \textit{inverse} does not exist. If the determinant is non-zero, the inverse of the matrix ($A^{-1}$) can be found with the formula
$$ A^{-1} = \dfrac{1}{\text{det}(A)} \times \begin{bmatrix} d & -b \\ -c & a \end{bmatrix} $$
	\begin{enumerate}
	\item Write a C program that will calculate the determinant of a $2\times2$ matrix.\\
	\item Expand on the previous program by calculating the inverse of the $2\times2$ matrix (only if it exists). If it does not exist, print a warning message to the user.
	\end{enumerate}


\end{task}

\end{document}