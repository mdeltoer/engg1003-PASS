\documentclass{pass}

\usepackage{graphicx}
\usepackage{float}
\usepackage{soul}
\usepackage{enumitem}
\usepackage{tabularx}
\usepackage{amsmath,amsfonts,amssymb} % For nice maths

\title{ENGG1003 - PASS Session 4}
\author{Mitchell Deltoer}
\date{Week 5}

\begin{document}
\begin{minipage}[H]{0.6\textwidth}
  \maketitle
\end{minipage}
\hfill
\begin{minipage}[H]{0.35\textwidth}
\textbf{
\begin{tabular}{ccc}
Monday & 14-15 & ES238 \\
Wednesday & 12-13 & ES238 \\
Thursday & 10-11 & MCLG42 \\
\end{tabular}
}
\end{minipage}

\section*{Switch, Case, Break - Optional}
A \textit{switch statement} is another kind of \textit{flow control} that allows a variable to be tested for equality against a list of values. Each value is called a \textit{case}, and the variable or expression being switched on is checked for each \textit{case}. The standard syntax of a \textit{switch statement} is
\begin{lstlisting}[style=CStyle]
switch(expression){
	// you can have any number of case statements
	case constant1:
		statement(s);
		break; // optional
	case constant2:
		statement(s);
		break; // optional
	default: // optional (this is executed if no other case is true)
	statement(s);
}
\end{lstlisting}
If the \texttt{break} statements are omitted, the program will continue executing each \textit{case} that follows the first valid check.

A \textit{switch statement} can be used in place of an \texttt{else if} chain for allowing a program to choose between several different options, especially where there is a large number of options to choose from. They are never absolutely necessary, it's your choice whether you include them in your program over an \texttt{else if} chain.
                                                                                                                                                                                       
\section*{Functions in C}
A \textit{function} is a block of code which can be called multiple times. Functions have a function name, an optional \textit{return value} (output) and optionally one or more \textit{arguments} (inputs). The standard structure of a \textit{function prototype} is
\begin{lstlisting}[style=CStyle]
	return_type function_name(arguments);
\end{lstlisting}
where \texttt{return\textunderscore type} is the data type of the \textit{return value} (\texttt{void} for no output), and  \texttt{arguments} is either \texttt{void} for no arguments or a list of arguments in the form 
\begin{lstlisting}[style=CStyle]
	arg1_type arg1_name, arg2_type arg2_name, ... , argN_type argN_name
\end{lstlisting}
where \texttt{arg\textunderscore type} is the data type of each argument and \texttt{arg\textunderscore name} is the name of each variable within the function.

Like variables, functions need to be defined before they are used. The \textit{function prototype} should be included before the \texttt{main} function. The \textit{function definition} (where the actual function code goes) should be included after the \texttt{main} function.

Functions are very useful when it comes to breaking down a problem into several sub-problems, and when a particular feature of your code is very cumbersome to read (more than 10-20 lines).

\section*{Variable Scope and Persistence}
Whenever a variable is declared within a \textit{code block} (between any pair of curly braces), i.e. in a \texttt{while} loop, \texttt{if} code block or a \textit{function}, the variable's ``existence" is limited to this code block. Where a variable exists is called a variable's \textit{scope}. 

Both the value and the definition of a variable are lost when the program is outside of a variable's scope. However, the value of a variable can be retained if it is declared as \texttt{static} but their \textit{scope} is still limited.

These concepts are critical for programming in C, especially when it comes to writing \textit{functions}.

\pagebreak

\section*{Practice Programming}
Some things to consider when using functions within your program:
\begin{itemize}
\item What \textit{arguments} (inputs) are there to the function? What are their data types?
\item What is the \textit{return value} (output) of the function? What is its data type?
\item Have I remembered to (correctly) include the \textit{function prototype} and the \textit{function definition}?
\end{itemize}

%A summary of some quick tips:
%\begin{itemize}
%\item \texttt{if} ... \texttt{else if} ... \texttt{else} statements are used to choose between several different options or cases.
%\item A \texttt{for} loop will execute a block of code a set number of times.
%\item A \texttt{while} loop can be used to execute a block of code either indefinitely, or it can be used like a \texttt{for} loop.\\[12pt]
%\end{itemize}

\begin{task}{Addition function}{}
Write a C program that reads two numbers and returns their sum. The addition should be done within a function called \texttt{addition}, and the main function should then print the result to the console window.
\end{task}

\begin{task}{Pythagoras function}{}
Write a C program that reads the lengths of two sides of a right-angled triangle and returns the hypotenuse of the triangle. The calculation should be done within a function called \texttt{pythagoras}, and the main function should then print the result to the console window.
\end{task}

\begin{task}{Maximum function}{}
Write a C program that reads two numbers and returns the largest number. The calculation should be done in a function called \texttt{maximum}, and the main function should then print the result to the console window.
\end{task}

\begin{task}{Factorial function}{}
Write a C program that reads an integer and returns the factorial of that number. The factorial should be computed within a function called \texttt{factorial}, and the main function should then print the result to the console window.
\end{task}

\begin{task}{Print function}{}
Write a C program that displays a block-letter of the first letter of your name using \texttt{*} characters. It should be printed through a function called \texttt{print\textunderscore letter}. Use the main function to call this function to view your result. The following shows a few examples of sample output (however, feel free to get creative).
\begin{lstlisting}[style=CStyle]
**    **     ******     ******      *****     **  **
***  ***       **         **       **         **  **
********       **         **       **         ******
** ** **       **         **       **         **  **
** ** **     ******       **        *****     **  **
\end{lstlisting}
\end{task}

\begin{task}{Temperature conversion function}{}
	\begin{enumerate}
	\item Write a C program that reads in a temperature in Celsius and returns the temperature in Fahrenheit. The conversion should be computed within a function called \texttt{temp\textunderscore convert}, and the main function should then print the result to the console window. \\[6pt]
	\boxed{F = \dfrac{9}{5}C + 32,  \hspace*{2cm} C = \dfrac{5}{9}(F-32)}
	\item Extend the previous program's functionality to instead read in a temperature in \textbf{either} Celsius or Fahrenheit, and also a character which will determine the units of their temperature input. The function should convert from C to F if their input is `C' or from F to C if the input is `F'. The main function should then print the result to the console window.
	\end{enumerate}
\end{task}

%\begin{task}{Modify previous labs}{}
%Try going back to any tasks done in previous weeks and see how you can modify them to include functions where appropriate.
%\end{task}

\end{document}